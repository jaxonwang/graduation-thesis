%% LyX 2.1.3 created this file.  For more info, see http://www.lyx.org/.
%% Do not edit unless you really know what you are doing.
\documentclass[oneside,english]{ctexbook}
\usepackage{fontspec}
\setcounter{secnumdepth}{3}
\setcounter{tocdepth}{3}

\makeatletter
%%%%%%%%%%%%%%%%%%%%%%%%%%%%%% User specified LaTeX commands.

\XeTeXlinebreaklocale "zh"
\XeTeXlinebreakskip = 0pt plus 1pt

\makeatother

\usepackage{xunicode}
\usepackage{polyglossia}
\setdefaultlanguage{english}
\begin{document}

\chapter{绪论}


\section{课题背景}

互联网的普及让人类进入了信息时代,各种各样的信息唾手可得。根据粗略估计,互联网每日产生的数据量已经达到EB的级别。这么多的数据导致了信息过载,用户面对这样庞大的数据,很难手动从中过滤掉无用的信息,筛选出有价值的信息。比如互联网电商,社交网站,新闻网站中,充斥着数以百万计的物品信息,对于用户来说,不可能完整地一一浏览过滤。针对数据过载这一问题,信息检索、数据挖掘、预测推荐等技术得到了广泛的应用。

搜索引擎是信息检索技术的集大成者。当今世界上搜索引擎Google一家独大,用户可以通过向搜索引擎提供关键词来获得搜索结果。同时,搜索引擎也可以通过数据挖掘技术为不同类别的信息分类或者提供标签。但是如果用户向搜索引擎提供的信息有限,甚至不知道什么才是自己所需要的,那么搜索引擎就不能很好地解决信息过载这一问题{[}1{]}。此时就需要能根据用户的喜好与品味给用户提供合适推荐的推荐系统了。

推荐系统又叫个性化推荐系统{[}2{]},是信息过滤系统的一个子集,它尝试预测用户可能对一个物品的“评分”和“喜好程度”{[}3{]}。推荐系统被证明是解决信息过载问题的一个有效工具,具体地来说,从用户的显式或者隐式信息中提取有效的部分,进行建模来分析用户的品味和喜好,再把用户可能喜欢的物品进行排序,对用户进行个性化的推荐。

推荐系统广泛地应用于互联网各大网站。通过推荐系统,能够很好地抓住用户的口味,符合用户的个性,大大提高了用户的转化率,为企业带来巨大的利润。下边介绍几种典型的应用领域。

\begin{itemize}
 \item wocao
\end{itemize}

一、电子商务 电子商务网站是推荐系统被广泛应用的一个邻域之一。亚马逊是著名的电子网站,它把推荐系统应用到了极致,也是最早应用推荐系统的网站之一。
它提出来名为item-to-item的协同过滤算法{[}4{]},推动了推荐系统邻域的发展。为后来用户登录后,它在首页为用户提供了个性化推荐列表。其中在每一样商品下方提供了其他商品的推荐,这些商品往往是其他用户最可能同时购买的,并对打包购买提供一定的折扣。
二、电影视频网站 在电影视频网站中,使用推荐系统能够帮助用户找到他们喜欢的视频。Youtube目前是全球最大的视频网站,拥有大量的用户数据已经大量的视频。Youtube通过应用推荐系统,大大增加了用户的视频点击率。根据实验,个性化推荐相对与非个性化推荐有两倍多点击率{[}5{]}。
三、音乐网络电台 音乐网络电台为用户提供了音乐点播服务,如果能准确地向用户推荐他喜爱的音乐那么会极大地增加用户的黏度。国内一个著名的音乐电台是douban.fm,
它为用户提供对歌曲喜欢或者删除的接口,收集用户对歌曲的反馈,从而提供准确的推荐。 四、社交网络 社交网络包含巨大的用户之间的社交信息。Facebook是社交网络的一个代表。它利用了用户的社交信息、偏好信息来推荐物品。Facebook对外提供了一个推荐API,可给用户推荐其好友喜欢的物品。
\end{document}
